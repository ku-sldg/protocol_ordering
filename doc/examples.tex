\newcolumntype{M}[1]{>{\centering\arraybackslash}m{#1}}
\newcolumntype{N}{@{}m{0pt}@{}}

We motivate this work specifically in the context of attack trees. 

\subsection{Chase Model Finder}

sys: *target: @p4 (vc p4 sys)

%\begin{figure}[htpb]
%    \centering 
%    \begin{tikzpicture}[->,>=stealth']

    \node[rectangle,
          draw,
          fill = green!30,
          minimum width = 2cm, 
          minimum height = 0.5cm
          ] (ms4) at (0,0) {};
    \node[] at (ms4.center) {ms4};


    \node[rectangle,
        draw,
        fill = yellow!30,
        minimum width = 1.5cm, 
        minimum height = 0.5cm
        ] (sys) at (-1,1) {};
    \node[] at (sys.center) {sys};

    \node[rectangle,
        draw,
        fill = yellow!30,
        minimum width = 1.5cm, 
        minimum height = 0.5cm
        ] (vc) at (1,1) {};
    \node[] at (vc.center) {vc};


    \path[every node/.style={font=\sffamily\small}]
    %host1 path
    (vc) edge [] node [right] {} (ms4.north) 
    (sys) edge [] node [right] {} (ms4.north) ;


\end{tikzpicture}
%    \label{fig:sys1}
%\end{figure}

\begin{figure}
\begin{center}
    \begin{tabular}{ M{3.75cm} | M{5cm} }
            sys & vc sys seq  \\
            \hline
            \hline
            \\ \begin{tikzpicture}[->,>=stealth']

    \node[rectangle,
          draw,
          fill = green!30,
          minimum width = 2cm, 
          minimum height = 0.5cm
          ] (ms4) at (0,0) {};
    \node[] at (ms4.center) {ms4};


    \node[rectangle,
        draw,
        fill = yellow!30,
        minimum width = 1.5cm, 
        minimum height = 0.5cm
        ] (sys) at (-1,1) {};
    \node[] at (sys.center) {sys};

    \node[rectangle,
        draw,
        fill = yellow!30,
        minimum width = 1.5cm, 
        minimum height = 0.5cm
        ] (vc) at (1,1) {};
    \node[] at (vc.center) {vc};


    \path[every node/.style={font=\sffamily\small}]
    %host1 path
    (vc) edge [] node [right] {} (ms4.north) 
    (sys) edge [] node [right] {} (ms4.north) ;


\end{tikzpicture} & \begin{tikzpicture}[->,>=stealth']

    \node[rectangle,
          draw,
          fill = green!30,
          minimum width = 2cm, 
          minimum height = 0.5cm
          ] (ms4) at (0,0) {};
    \node[] at (ms4.center) {ms4};


    \node[rectangle,
        draw,
        fill = yellow!30,
        minimum width = 1.5cm, 
        minimum height = 0.5cm
        ] (sys) at (-1,1) {};
    \node[] at (sys.center) {sys};

    \node[rectangle,
        draw,
        fill = yellow!30,
        minimum width = 1.5cm, 
        minimum height = 0.5cm
        ] (vc) at (1,1) {};
    \node[] at (vc.center) {vc};

    \node[rectangle,
        draw,
        minimum width = 1.5cm, 
        minimum height = 0.5cm
        ] (ms3) at (1,2) {};
    \node[] at (ms3.center) {ms3};


    \path[every node/.style={font=\sffamily\small}]
    %host1 path
    (ms3) edge [] node [right] {} (vc.north)
    (vc) edge [] node [right] {} (ms4.north) 
    (sys) edge [] node [right] {} (ms4.north) ;


\end{tikzpicture}  \\ [50pt]
            \\ \begin{tikzpicture}[->,>=stealth']

    \node[rectangle,
          draw,
          fill = green!30,
          minimum width = 2cm, 
          minimum height = 0.5cm
          ] (ms4) at (0,0) {};
    \node[] at (ms4.center) {ms4};


    \node[rectangle,
        draw,
        fill = yellow!30,
        minimum width = 1.5cm, 
        minimum height = 0.5cm
        ] (sys) at (-1,1) {};
    \node[] at (sys.center) {sys};

    \node[rectangle,
        draw,
        fill = yellow!30,
        minimum width = 1.5cm, 
        minimum height = 0.5cm
        ] (ker) at (1,1) {};
    \node[] at (ker.center) {ker};


    \path[every node/.style={font=\sffamily\small}]
    %host1 path
    (vc) edge [] node [right] {} (ms4.north) 
    (sys) edge [] node [right] {} (ms4.north) ;


\end{tikzpicture} & \begin{tikzpicture}[->,>=stealth']

    \node[rectangle,
          draw,
          fill = green!30,
          minimum width = 2cm, 
          minimum height = 0.5cm
          ] (ms4) at (0,0) {};
    \node[] at (ms4.center) {ms4};


    \node[rectangle,
        draw,
        fill = yellow!30,
        minimum width = 1.5cm, 
        minimum height = 0.5cm
        ] (sys) at (-1,1) {};
    \node[] at (sys.center) {sys};

    \node[rectangle,
        draw,
        fill = yellow!30,
        minimum width = 1.5cm, 
        minimum height = 0.5cm
        ] (vc) at (1,1) {};
    \node[] at (vc.center) {vc};

    \node[rectangle,
        draw,
        minimum width = 1.5cm, 
        minimum height = 0.5cm
        ] (ms3) at (0,2) {};
    \node[] at (ms3.center) {ms3};


    \path[every node/.style={font=\sffamily\small}]
    %host1 path
    (ms3) edge [] node [right] {} (ms4.north)
    (vc) edge [] node [right] {} (ms4.north) 
    (sys) edge [] node [right] {} (ms4.north) ;


\end{tikzpicture}  \\ [50pt]  
             & \begin{tikzpicture}[->,>=stealth']

    \node[rectangle,
          draw,
          fill = green!30,
          minimum width = 2cm, 
          minimum height = 0.5cm
          ] (ms4) at (0,0) {};
    \node[] at (ms4.center) {ms4};


    \node[rectangle,
        draw,
        minimum width = 1.5cm, 
        minimum height = 0.5cm
        ] (ms3) at (-1,1) {};
    \node[] at (ms3.center) {ms3};

    \node[rectangle,
        draw,
        fill = yellow!30,
        minimum width = 1.5cm, 
        minimum height = 0.5cm
        ] (sys) at (1,1) {};
    \node[] at (sys.center) {sys};

    \node[rectangle,
        draw,
        fill = yellow!30,
        minimum width = 1.5cm, 
        minimum height = 0.5cm
        ] (vc) at (-2,2) {};
    \node[] at (vc.center) {vc};

    \node[rectangle,
        draw,
        fill = yellow!30,
        minimum width = 1.5cm, 
        minimum height = 0.5cm
        ] (a) at (0,2) {};
    \node[] at (a.center) {a};


    \path[every node/.style={font=\sffamily\small}]
    %host1 path
    (a) edge [] node [right] {} (ms3.north)
    (vc) edge [] node [right] {} (ms3.north)
    (sys) edge [] node [right] {} (ms4.north) 
    (ms3) edge [] node [right] {} (ms4.north) ;


\end{tikzpicture}  \\ [50pt]  
            \\  & \begin{tikzpicture}[->,>=stealth']

    \node[rectangle,
          draw,
          fill = green!30,
          minimum width = 2cm, 
          minimum height = 0.5cm
          ] (ms4) at (0,0) {};
    \node[] at (ms4.center) {ms4};


    \node[rectangle,
        draw,
        minimum width = 1.5cm, 
        minimum height = 0.5cm
        ] (ms3) at (-1,1) {};
    \node[] at (ms3.center) {ms3};

    \node[rectangle,
        draw,
        fill = yellow!30,
        minimum width = 1.5cm, 
        minimum height = 0.5cm
        ] (sys) at (1,1) {};
    \node[] at (sys.center) {sys};

    \node[rectangle,
        draw,
        fill = yellow!30,
        minimum width = 1.5cm, 
        minimum height = 0.5cm
        ] (vc) at (-2,2) {};
    \node[] at (vc.center) {vc};

    \node[rectangle,
        draw,
        fill = yellow!30,
        minimum width = 1.5cm, 
        minimum height = 0.5cm
        ] (a) at (0,2) {};
    \node[] at (a.center) {c};


    \path[every node/.style={font=\sffamily\small}]
    %host1 path
    (a) edge [] node [right] {} (ms3.north)
    (vc) edge [] node [right] {} (ms3.north)
    (sys) edge [] node [right] {} (ms4.north) 
    (ms3) edge [] node [right] {} (ms4.north) ;


\end{tikzpicture}  \\ [50pt] 
        \end{tabular}
    \end{center}
    \caption{All attack graphs for two protocols}
\end{figure}


\begin{figure}
    \begin{center}
        \begin{tabular}{ M{3.75cm} | M{3.75cm} | M{3.75cm} }
                sys & vc sys seq & ker vc sys seq \\
                \hline
                \hline
                \\ \begin{tikzpicture}[->,>=stealth']

    \node[rectangle,
          draw,
          fill = green!30,
          minimum width = 2cm, 
          minimum height = 0.5cm
          ] (ms4) at (0,0) {};
    \node[] at (ms4.center) {ms4};


    \node[rectangle,
        draw,
        fill = yellow!30,
        minimum width = 1.5cm, 
        minimum height = 0.5cm
        ] (sys) at (-1,1) {};
    \node[] at (sys.center) {sys};

    \node[rectangle,
        draw,
        fill = yellow!30,
        minimum width = 1.5cm, 
        minimum height = 0.5cm
        ] (vc) at (1,1) {};
    \node[] at (vc.center) {vc};


    \path[every node/.style={font=\sffamily\small}]
    %host1 path
    (vc) edge [] node [right] {} (ms4.north) 
    (sys) edge [] node [right] {} (ms4.north) ;


\end{tikzpicture} & \begin{tikzpicture}[->,>=stealth']

    \node[rectangle,
          draw,
          fill = green!30,
          minimum width = 2cm, 
          minimum height = 0.5cm
          ] (ms4) at (0,0) {};
    \node[] at (ms4.center) {ms4};


    \node[rectangle,
        draw,
        fill = yellow!30,
        minimum width = 1.5cm, 
        minimum height = 0.5cm
        ] (sys) at (-1,1) {};
    \node[] at (sys.center) {sys};

    \node[rectangle,
        draw,
        fill = yellow!30,
        minimum width = 1.5cm, 
        minimum height = 0.5cm
        ] (vc) at (1,1) {};
    \node[] at (vc.center) {vc};

    \node[rectangle,
        draw,
        minimum width = 1.5cm, 
        minimum height = 0.5cm
        ] (ms3) at (1,2) {};
    \node[] at (ms3.center) {ms3};


    \path[every node/.style={font=\sffamily\small}]
    %host1 path
    (ms3) edge [] node [right] {} (vc.north)
    (vc) edge [] node [right] {} (ms4.north) 
    (sys) edge [] node [right] {} (ms4.north) ;


\end{tikzpicture} & cell3 \\ [50pt]
                \\ \begin{tikzpicture}[->,>=stealth']

    \node[rectangle,
          draw,
          fill = green!30,
          minimum width = 2cm, 
          minimum height = 0.5cm
          ] (ms4) at (0,0) {};
    \node[] at (ms4.center) {ms4};


    \node[rectangle,
        draw,
        fill = yellow!30,
        minimum width = 1.5cm, 
        minimum height = 0.5cm
        ] (sys) at (-1,1) {};
    \node[] at (sys.center) {sys};

    \node[rectangle,
        draw,
        fill = yellow!30,
        minimum width = 1.5cm, 
        minimum height = 0.5cm
        ] (ker) at (1,1) {};
    \node[] at (ker.center) {ker};


    \path[every node/.style={font=\sffamily\small}]
    %host1 path
    (vc) edge [] node [right] {} (ms4.north) 
    (sys) edge [] node [right] {} (ms4.north) ;


\end{tikzpicture} & \begin{tikzpicture}[->,>=stealth']

    \node[rectangle,
          draw,
          fill = green!30,
          minimum width = 2cm, 
          minimum height = 0.5cm
          ] (ms4) at (0,0) {};
    \node[] at (ms4.center) {ms4};


    \node[rectangle,
        draw,
        fill = yellow!30,
        minimum width = 1.5cm, 
        minimum height = 0.5cm
        ] (sys) at (-1,1) {};
    \node[] at (sys.center) {sys};

    \node[rectangle,
        draw,
        fill = yellow!30,
        minimum width = 1.5cm, 
        minimum height = 0.5cm
        ] (vc) at (1,1) {};
    \node[] at (vc.center) {vc};

    \node[rectangle,
        draw,
        minimum width = 1.5cm, 
        minimum height = 0.5cm
        ] (ms3) at (0,2) {};
    \node[] at (ms3.center) {ms3};


    \path[every node/.style={font=\sffamily\small}]
    %host1 path
    (ms3) edge [] node [right] {} (ms4.north)
    (vc) edge [] node [right] {} (ms4.north) 
    (sys) edge [] node [right] {} (ms4.north) ;


\end{tikzpicture} & cell6 \\ [50pt]  
                 & \begin{tikzpicture}[->,>=stealth']

    \node[rectangle,
          draw,
          fill = green!30,
          minimum width = 2cm, 
          minimum height = 0.5cm
          ] (ms4) at (0,0) {};
    \node[] at (ms4.center) {ms4};


    \node[rectangle,
        draw,
        minimum width = 1.5cm, 
        minimum height = 0.5cm
        ] (ms3) at (-1,1) {};
    \node[] at (ms3.center) {ms3};

    \node[rectangle,
        draw,
        fill = yellow!30,
        minimum width = 1.5cm, 
        minimum height = 0.5cm
        ] (sys) at (1,1) {};
    \node[] at (sys.center) {sys};

    \node[rectangle,
        draw,
        fill = yellow!30,
        minimum width = 1.5cm, 
        minimum height = 0.5cm
        ] (vc) at (-2,2) {};
    \node[] at (vc.center) {vc};

    \node[rectangle,
        draw,
        fill = yellow!30,
        minimum width = 1.5cm, 
        minimum height = 0.5cm
        ] (a) at (0,2) {};
    \node[] at (a.center) {a};


    \path[every node/.style={font=\sffamily\small}]
    %host1 path
    (a) edge [] node [right] {} (ms3.north)
    (vc) edge [] node [right] {} (ms3.north)
    (sys) edge [] node [right] {} (ms4.north) 
    (ms3) edge [] node [right] {} (ms4.north) ;


\end{tikzpicture} & cell9 \\ [50pt]  
                \\  & \begin{tikzpicture}[->,>=stealth']

    \node[rectangle,
          draw,
          fill = green!30,
          minimum width = 2cm, 
          minimum height = 0.5cm
          ] (ms4) at (0,0) {};
    \node[] at (ms4.center) {ms4};


    \node[rectangle,
        draw,
        minimum width = 1.5cm, 
        minimum height = 0.5cm
        ] (ms3) at (-1,1) {};
    \node[] at (ms3.center) {ms3};

    \node[rectangle,
        draw,
        fill = yellow!30,
        minimum width = 1.5cm, 
        minimum height = 0.5cm
        ] (sys) at (1,1) {};
    \node[] at (sys.center) {sys};

    \node[rectangle,
        draw,
        fill = yellow!30,
        minimum width = 1.5cm, 
        minimum height = 0.5cm
        ] (vc) at (-2,2) {};
    \node[] at (vc.center) {vc};

    \node[rectangle,
        draw,
        fill = yellow!30,
        minimum width = 1.5cm, 
        minimum height = 0.5cm
        ] (a) at (0,2) {};
    \node[] at (a.center) {c};


    \path[every node/.style={font=\sffamily\small}]
    %host1 path
    (a) edge [] node [right] {} (ms3.north)
    (vc) edge [] node [right] {} (ms3.north)
    (sys) edge [] node [right] {} (ms4.north) 
    (ms3) edge [] node [right] {} (ms4.north) ;


\end{tikzpicture} & cell6 \\ [50pt] 
            \end{tabular}
        \end{center}
        \caption{All attack graphs for two protocols}
    \end{figure}