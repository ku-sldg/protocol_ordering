\section{Mathematical Concepts }

\begin{definition}[Homomorphism]
    A (graph) homomorphism $\eta : G \to H$ between graphs $G = (N_G, E_G, l_G)$ and $H = (N_H, E_H, l_H)$ is a function $\eta : N_G \to N_H$ on the nodes such that for every edge $(n_1, n_2) \in E_G$, $(\eta(n_1), \eta(n_2)) \in E_H$, and for every node $n \in N_G$, $l_G(n) = l_H(n)$.  \cite{Rowe:2021:OnOrdering}
\end{definition}

A homomorphism is a relation between two graphs, $G$ and $H$, where two graphs are homomorphic if there exists some function which maps the nodes of $G$ to the nodes of $H$ such that all edges in $G$ have a corresponding edge in $H$ and all labels in $G$ have related labels in $H$. To prove two graphs are homomorphic, one must prove there exists a function not necessarily that every function produces a homomorphism. Therefore, to prove a homomorphism does not exists, one must prove no functions exists. 

\begin{definition}[Supports/Covers]
    Given two sets of graphs S and T, we say that S supports T iff for every $H \in T$, there is some $G \in S$, such that $G \leq H$. We  say that T covers S iff for every $G \in S$ there is some $H \in T$ such that $G \leq H. $\cite{Rowe:2021:OnOrdering}
\end{definition}

Rowe defines supports and covers to introduce a relation over sets of graphs. Supports states that, for every graph in $T$, there exists some homomorphism from a graph in $S$ to a graph in $T$. Conversely, covers says that, for every graph in $S$, there exists a homomorphism to some graph in T. We imagine we can apply the ideas of supports and covers to compare sets of graphs using relations besides homomorphisms. 

Through our study, we found that a homomorphism was not the correct way to capture the relationship between attack graphs. 

