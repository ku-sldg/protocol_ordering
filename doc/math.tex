\section{Mathematical Concepts}

\subsection{Basics}

Critical to our study is various relations $R$ on a set $X$. We specifically consider the following relations: 

\begin{itemize}
    \item \emph{reflexive} -- forall $ x \in X$, $R$ x x
    \item \emph{symmetric} -- forall $ x , y\in X$, $R$ x y $\rightarrow$ $R$ y x
    \item \emph{transitive} -- forall $ x , y , z\in X$, $R$ x y $\rightarrow$ $R$ y z $\rightarrow$ $R$ x z
    \item \emph{antisymmetric} -- forall $ x , y\in X$, $R$ x y \& $R$ y x $\rightarrow$ x = y
    \item an \emph{equivalence} relation is reflexive, symmetric, and transitive 
    \item a \emph{partial order} is reflexive, antisymmetric, and transitive 
    \item a \emph{preorder} is reflexive and transitive
\end{itemize}

We also may need to consider properties of functions. A function from set $X$ to set $Y$ is written $f : X \rightarrow Y$. Here $X$ is the domain while $Y$ is the codomain. A function may also be considered a relation for pairs $(x,y)$ where $f(x) = y$. \emph{Partial} functions may be undefined for elements in the domain while a \emph{total} function is defined for all elements of the domain. We can say various things about properties of functions. We say a function $f : X \rightarrow Y$ is 

\begin{itemize}
    \item \emph{injective} -- forall $ x , y \in X$, $f(x) = f(y)$ $\rightarrow$  $x= y$
    \item \emph{surjective} -- forall $y \in Y$, there exists some $x \in X$ with  $f(x) = y$
    \item a \emph{bijective} function is both injective (one-to-one) and surjective (onto) 
\end{itemize}

\subsection{Mathematical Structures}

% Graphs 
In this work, we are mostly concerned with directed, labeled, acyclic graphs as defined below. 

\begin{definition}[Directed graph]
    A directed, labeled, acyclic graph is a tuple $ G = (N, E, \ell)$ where $N$ is a finite set of nodes, $E$ is a finite set of edges represented as an edge relation $( N \times N)$, and $\ell$ is an node labeling function $\ell : N \rightarrow L$. 
\end{definition}

% Transition systems 
\begin{definition}[Transition System]
    A transition system is a pair $(S,T)$ where $S$ is the set of states and $T$ is the transition relation between states $ ( S \times S )$. We say some state $p$ transitions to $q$ if and only if $(p,q) \in T$. \\ The transition from $p$ to $q$ is denoted $ p \rightarrow q $.
\end{definition}

One may easily see that transition systems resemble graphs. In this way, we may say that two transition systems are equivalence if they are isomorphic (there exists a bijection between the states and transitions).  
\todo{what difference between directed graphs and labeled transition systems?} 

If the transitions have labeled, then the transition system becomes a labeled transition system. 

% Labeled transition systems 
\begin{definition}[Labeled Transition System]
    A labeled transition system is a tuple $(S, L, T)$ where $S$ is the set of states, $L$ is the set of labels, and $T$ is the transition relation between states $ ( S \times L \times S )$. We say some state $p$ transitions to $q$ with label $a$ if and only if $(p,a,q) \in T$. \\ The transition from $p$ to $q$ with label $a$ is denoted $ p \xrightarrow{a} q $.
\end{definition}

A Kripke structure is another variation of transition systems. In this model, the nodes are reachable states and the edges are the state transitions. The labeleing function of a Kripke structure maps the nodes to the set of properties that hold within the state. 

\begin{definition}[Kripke Structure]
    A Kripke structure is a 4-tuple $(S, I, L, T)$ where $S$ is the set of states, $I$ is the set of initial states, $L$ is a labeling from $S \rightarrow Prop$,  and $T$ is the transition relation between states $ ( S \times S )$. The transition relation is left-total meaning $\forall s \in S, \exists s' \in S$ such that $(s,s') \in R$ 
    
\end{definition}

\subsection{Properties over structures}

\begin{definition}[Homomorphism]
    A (graph) homomorphism $\eta : G \to H$ between graphs $G = (N_G, E_G, l_G)$ and $H = (N_H, E_H, l_H)$ is a function $\eta : N_G \to N_H$ on the nodes such that for every edge $(n_1, n_2) \in E_G$, $(\eta(n_1), \eta(n_2)) \in E_H$, and for every node $n \in N_G$, $l_G(n) = l_H(n)$.  \cite{Rowe:2021:OnOrdering}
\end{definition}

A homomorphism is a relation between two graphs, $G$ and $H$, where two graphs are homomorphic if there exists some function which maps the nodes of $G$ to the nodes of $H$ such that all edges in $G$ have a corresponding edge in $H$ and all labels in $G$ have related labels in $H$. To prove two graphs are homomorphic, one must prove there exists a function not necessarily that every function produces a homomorphism. Therefore, to prove a homomorphism does not exists, one must prove no functions exists. 

A homomorphism bestows a preorder i.e. it is reflexive and transitive. If the homomorphism is injective, then the preorder is a partial order (up to an isomorphism). This is because injective homomorphisms imply an isomorphism. 

\begin{definition}[Supports/Covers]
    Given two sets of graphs S and T, we say that S supports T iff for every $H \in T$, there is some $G \in S$, such that $G \leq H$. We  say that T covers S iff for every $G \in S$ there is some $H \in T$ such that $G \leq H. $\cite{Rowe:2021:OnOrdering}
\end{definition}

Rowe defines supports and covers to introduce a relation over sets of graphs. Supports states that, for every graph in $T$, there exists some homomorphism from a graph in $S$ to a graph in $T$. Conversely, covers says that, for every graph in $S$, there exists a homomorphism to some graph in T. We imagine we can apply the ideas of supports and covers to compare sets of graphs using relations besides homomorphisms. 

Through our study, we found that a homomorphism was not the correct way to capture the relationship between attack graphs. 

% Bisimulation

Bisimulation is a means of determining process equivalence \cite{sangiorgi_2011}. With this, we can state two processes are equal if they have the same behavior. 

\begin{definition}[Bisimulation]
    A relation $R$ is a bisimulation if, whenever P $\Re$ Q, for all $\mu$ we have: 
    \begin{itemize}
        \item forall $P'$ with $P \rightarrow P'$, exists $Q'$ such that Q $\rightarrow$ $Q'$ and P' $\Re$ Q'
        \item forall $Q'$ with $Q \rightarrow Q'$, exists $P'$ such that P $\rightarrow$ $P'$ and P' $\Re$ Q'
    \end{itemize}
    
\end{definition}

% properties of bisimulation 

% Variants of bisimulation -- weak bisimulation 

\begin{definition}[Weak Bisimulation]
    
\end{definition}



